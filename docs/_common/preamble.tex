% ---------------------------------------------------------------------------- %
%                                   PACKAGES                                   %
% ---------------------------------------------------------------------------- %

% --- 한글 및 기본 설정 패키지 ---
\usepackage{kotex}                     % 한글 지원
\usepackage{graphicx}                  % 그림 포함
\usepackage{xcolor}                    % 색상 지원

% --- 제목/번호/목차 관련 ---
\usepackage{titling}                   % 커스텀 표지 지원
\usepackage{titlesec}                  % 제목 포맷 커스텀
\usepackage{setspace}                  % 줄간격 조정
\setstretch{1.5}                       % 기본 줄간격 1.5배
\setlength{\parskip}{0.6em}            % 단락 사이 간격
\setlength{\parindent}{1em}            % 단락 들여쓰기

% --- 수식 및 수학 패키지 ---
\usepackage{amsmath}                   % 수학 공식
\numberwithin{equation}{chapter}       % 챕터별 수식번호

% --- 표, 그림 관련 패키지 ---
\usepackage{booktabs}                  % \toprule 등 고급 표
\usepackage{longtable}                 % 페이지 나누는 표
\usepackage{tabularx}                  % 가변 폭 표
\usepackage{multirow}                  % 셀 병합
\usepackage{makecell}
\usepackage[tableposition=top]{caption}% 표 캡션 위 고정
\usepackage{subcaption}
\usepackage{array}
\newcolumntype{Y}{>{\centering\arraybackslash}X}

% --- 기타 디자인 ---
\usepackage[most]{tcolorbox}                 % 박스 디자인

% --- 레이아웃, 그림폴더 등 ---
\usepackage[margin=2cm,headheight=1cm]{geometry}
\graphicspath{{./_figures/}}

% --- 코드류 ---
\usepackage{listings}
\usepackage{ifthen}

% --- 주석 --- 
\usepackage{todonotes}

% --- 링크 ---
\usepackage[colorlinks, linkcolor=blue, citecolor=red]{hyperref} % 하이퍼링크
\usepackage[capitalise]{cleveref}      % cref 레퍼런스
\hypersetup{
  pdftitle={시뮬레이터 기술문서},
  pdfauthor={서울대학교 건축에너지연구실},
  pdfkeywords={},
  bookmarksopen=true
}
% ---------------------------------------------------------------------------- %
%                                   SETTINGS                                   %
% ---------------------------------------------------------------------------- %

% --- 목차, 그림/표목차, 참고문헌 등 한글화 ---
\renewcommand{\contentsname}{목차}
\renewcommand{\listfigurename}{그림 목록}
\renewcommand{\listtablename}{표 목록}
\renewcommand{\bibname}{참고문헌}

% --- 그림, 표, 식, 장, 절 인용 표기 --- 
\crefname{figure}{그림}{그림}
\Crefname{figure}{그림}{그림}
\crefname{table}{표}{표}
\Crefname{table}{표}{표}
\crefname{equation}{식}{식}
\Crefname{equation}{식}{식}
% \crefname{chapter}{장}{장}  % 밑에 crefformat이 오버라이딩해서 없어도 될듯
\crefformat{chapter}{#2#1장#3}
\crefformat{section}{#2#1절#3}
\crefformat{subsection}{#2#1절#3}

% --- secnum, 제목 스타일, 단락 스타일 등 ---
\setcounter{secnumdepth}{3}            % subsubsection까지 번호 매김

% --- part 스타일 ---
\titleformat{\part}
  {\centering\Huge\bfseries}
  {\thepart}
  {0pt}
  {
    \thispagestyle{empty}
  }

% --- chapter 스타일: 본문과 appendix에 따라 변경 가능하게 ---
\newcommand{\chapterlabel}{제\thechapter 장}
\titleformat{\chapter}[hang]
  {\normalfont\huge\bfseries}
  {\chapterlabel}
  {1em}
  {}

% --- paragraph, subparagraph 스타일 ---
\titleformat{\paragraph}[runin]
  {\normalfont\itshape\bfseries}
  {\theparagraph}{1em}{}
\titleformat{\subparagraph}[runin]
  {\normalfont\itshape\bfseries}
  {\thesubparagraph}{1em}{}

% --- 메모 스타일 ---
\definecolor{gonie_color}{RGB}{255, 255, 180}  
\definecolor{hongI_color}{RGB}{200, 255, 200} 
\definecolor{theungDu_color}{RGB}{176, 224, 230}   
\definecolor{형진_color}{RGB}{255, 220, 230}   

\let\originaltodo\todo
\renewcommand{\todo}[2]{%
  \ifthenelse{\equal{#1}{형곤}}%
    {\originaltodo[color=gonie_color, size=tiny]{\textbf{#1}: #2}}%
  {%
  \ifthenelse{\equal{#1}{철홍}}%
    {\originaltodo[color=hongI_color, size=tiny]{\textbf{#1}: #2}}%
  {%
  \ifthenelse{\equal{#1}{승주}}%
    {\originaltodo[color=theungDu_color, size=tiny]{\textbf{#1}: #2}}%
  {%
  \ifthenelse{\equal{#1}{형진}}%
    {\originaltodo[color=형진_color, size=tiny]{\textbf{#1}: #2}}%
  {%
    \originaltodo[color=gray!20, size=tiny]{\textbf{#1}: #2} % 기본값
  }}}}%
}
% ---------------------------------------------------------------------------- %
%                                     STYLE                                    %
% ---------------------------------------------------------------------------- %

% --- 그림, 표 환경 커스텀 ---
\newenvironment{defaultfigure}
  {%
  \vspace{0.3em}
  \begin{figure}[htbp]\centering
    \setlength{\belowcaptionskip}{0.3em}
  }
  {\end{figure}}

\renewcommand{\arraystretch}{1.5}
\newenvironment{defaulttable}
  {\begin{table}[htbp]\centering\small}
  {\end{table}}

\newenvironment{variabletable}
  {\begin{defaulttable}}
  {\end{defaulttable}}

% 단위 표시
\newcommand{\unitsty}[1]{{\small $#1$}}
% ---------------------------------------------------------------------------- %
%                                     NAMES                                    %
% ---------------------------------------------------------------------------- %

% --- release 관련 정보 (변수형) ---
\newcommand{\releaseversion}{0-4-5-1}
\newcommand{\releasedate}{2025.09.25.}


% --- 반복되는 용어/alias 정의 ---
\newcommand{\ep}{EnergyPlus}
\newcommand{\simulator}{epGRsim}

\newcommand{\pymodule}{pyGRsim}
\newcommand{\unit}[1]{
  \begingroup
  \ifthenelse{\equal{#1}{density}}{$kg/m^3$}{%
  \ifthenelse{\equal{#1}{conductivity}}{$W/mK$}{%
  \ifthenelse{\equal{#1}{U}}{$W/m^2K$}{-}}}
  \endgroup
}

% --- 