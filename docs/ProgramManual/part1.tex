% chap1.tex
\part{개요}
\label{part:introduction}

% ---------------------------------------------------------------------------- %
%                                  NEW SECTION                                 %
% ---------------------------------------------------------------------------- %

\chapter{개요}
아니 뭔가 이상한데? 왜냐면
챕터 밑에 section이 있고 그밑에 ?? 또 ?? 
본 프로그램매뉴얼은 \fullnamepolicy (이하 ``\policy"\라 한다) 에 따라 그린리모델링을 위한 의사결정시.. 변수의 입력에 대한 가이드라인을 제시하는 것을 목적으로 한다.

\section{대상}
본 가이드라인은 기축 건물 중 그린리모델링 대상지의 건물 속성 정보들을 다룬다. 건물 평면도, 단면도, 기계 및 전기 설비 도면을 바탕으로 시뮬레이터 변수 입력요령 등을 기술하고, 관련 사례 사진을 함께 첨부하여 활용할 수 있도록 하였다.

% ---------------------------------------------------------------------------- %
%                                  NEW SECTION                                 %
% ---------------------------------------------------------------------------- %

\section{설치 방법}
\subsection{다운로드}
\simulator\는 국토안전관리원 홈페이지에서 설치파일을 다운로드 할 수 있다.

\subsection{설치}
설치를 하고 나면은..

% ---------------------------------------------------------------------------- %
%                                  NEW SECTION                                 %
% ---------------------------------------------------------------------------- %

\section{\simulator의 접근 및 철학}

\subsection{원칙}

ECO2만큼의 노력으로 high-fidelity 동적 시뮬레이션 가능

\subsection{다른 도구들과의 관계}
\subsubsection{EnergyPlus, DesignBuilder}
솔직히 쓰기 힘듦

\subsubsection{ClimateStudio, Honeybee}
구조적으로는 얘를 많이 따 왔음.

\subsubsection{ECO2}
시뮬레이터는 ECO2의 입력 수준을 거의 그대로 유지하며 개발되었음.

\subsubsection{trace700}
조사 어려우면 삭제