\documentclass[a4paper,11pt]{article}

% ======================
% 한글 처리 (유일하게 필요한 패키지)
% ======================
\usepackage{kotex}   % ✅ UTF-8 + 한글 조합 + 폰트 자동 선택

% ======================
% 일반 설정
% ======================
\usepackage{graphicx}
\usepackage{booktabs}
\usepackage{subcaption}
\usepackage{geometry}
\geometry{margin=2.5cm}
\usepackage{setspace}
\usepackage{caption}
\usepackage{titlesec}
\setstretch{1.25}


\graphicspath{{figures/}}


\titleformat{\section}{\large\bfseries}{\thesection.}{0.5em}{}
\titleformat{\subsection}{\normalsize\bfseries}{\thesubsection}{0.5em}{}

% ======================
% 문서 본문
% ======================
\title{건물 리모델링 성능 보고서}
\author{자동 생성 시스템 (EplusSimple)}
\date{\today}

\begin{document}
\maketitle

\section{기상 데이터 비교}
{{ weather_summary }}

\begin{figure}[h]
\centering
\includegraphics[width=0.8\textwidth]{ {{ weather_fig }} }
\caption{세 기간별 월평균 외기온도 비교.}
\end{figure}

\section{건물 성능 비교}
\begin{table}[h]
\centering
\caption{단열 및 창호 성능 비교}
\begin{tabular}{lccc}
\toprule
항목 & 리모델링 전 & 리모델링 후 & N년차 \\
\midrule

{{ row.name }} & {{ row.before }} & {{ row.after }} & {{ row.nyear }} \\

\bottomrule
\end{tabular}
\end{table}

\section{에너지 시뮬레이션 결과}
\begin{figure}[h]
\centering
\begin{subfigure}{0.48\textwidth}
\includegraphics[width=\textwidth]{ {{ energy_fig1 }} }
\caption{연간 에너지 사용량 비교}
\end{subfigure}
\begin{subfigure}{0.48\textwidth}
\includegraphics[width=\textwidth]{ {{ energy_fig2 }} }
\caption{월별 에너지 사용 경향}
\end{subfigure}
\end{figure}

\section{결론}
{{ conclusion_text }}

\vfill
\noindent\rule{\textwidth}{0.4pt}\\[-4pt]
{\small 생성일: {{ today }}}
\end{document}
