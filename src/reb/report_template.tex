\documentclass[a4paper,10pt]{article}

% ======================
% 한글 처리 (유일하게 필요한 패키지)
% ======================
\usepackage{kotex}   % ✅ UTF-8 + 한글 조합 + 폰트 자동 선택
\renewcommand{\familydefault}{\sfdefault}

% ======================
% 일반 설정
% ======================
\usepackage{graphicx}
\usepackage{booktabs}
\usepackage{geometry}
\geometry{margin=2.5cm}
\usepackage{setspace}
\usepackage{caption}
\usepackage{titlesec}
\usepackage{xcolor}
\usepackage{colortbl}
\setstretch{1.25}
\usepackage[most]{tcolorbox}


\graphicspath{{figures/}}


\titleformat{\section}{\large\bfseries}{\thesection.}{0.5em}{}
\titleformat{\subsection}{\normalsize\bfseries}{\thesubsection}{0.5em}{}
% \section 번호 대신 네모 기호
\titleformat{\section}
  {\large\bfseries}   % 글꼴 설정
  {\makebox[1.5em][c]{\rule{0.8em}{0.8em}}}  % <-- 네모 (검정색 사각형)
  {0.5em}{}           % 간격

% ======================
% 문서 본문
% ======================
\title{건물 리모델링 성능 보고서}

\begin{document}
\begin{center}
    \Large\bfseries GR 정성평가: {{metadata.name}}
\end{center}

% ==================== 갑지 ====================

\section{건물 개요}

\newcolumntype{L}{>{\columncolor[gray]{0.9}\bfseries\centering\arraybackslash}p{0.25\textwidth}}
\newcolumntype{C}{>{\centering\arraybackslash}p{0.65\textwidth}}

% 본문
\setlength{\tabcolsep}{0pt}
\renewcommand{\arraystretch}{1.3}

\begin{table}[h!]
\centering
% 양쪽 끝 세로선은 빼고(요구사항), 가운데만 세로선
\begin{tabular}{L|C}
\hline
면적 & {{metadata.area}}$\mathrm{m^2}$ \\ \hline
허가연도 & {{metadata.date}} \\ \hline
주소(시군구) & {{metadata.addr}} \\ \hline
\end{tabular}
\end{table}

\section{리모델링 및 운영특성 변경 내역 요약}

\section{시뮬레이션 결과}
\begin{figure}[h!]
    \centering
    \begin{minipage}[b]{0.49\textwidth}
        \includegraphics[width=\textwidth]{energy_summary_use.png}
    \end{minipage}
    \hfill
    \begin{minipage}[b]{0.49\textwidth}
        \includegraphics[width=\textwidth]{energy_summary_co2.png}
    \end{minipage}
\end{figure}


\section{결론}
\begin{tcolorbox}[colback=gray!5,colframe=black!60,boxrule=0.6pt,
                  left=4mm,right=4mm,top=2mm,bottom=2mm]
\textbf{ {{metadata.name}} }은,

① 성능 변화 및 기상데이터의 변화를 고려할 때, \textbf{\underline{${{EUIdiff[0]}}\,\mathrm{kWh/m^2}$}} 절감될 것으로 분석되었다.

② 운영특성 및 추가적인 성능 변화를 고려할 때, 
\underline{${{EUIdiff[1]}}\,\mathrm{kWh/m^2}$} 절감될 것으로 분석되었다. 

③ 이는 운영특성 및 성능 변화로 그린리모델링 효과가 
\underline{${{EUIdiff[2]}}\,\mathrm{kWh/m^2}$} 줄어든 것이다.
\end{tcolorbox}

% ==================== 을지 ====================

\newpage
\section{성능 변화 내역 (GR 이전 → GR 이후)}

\section{기상 변화 내역 (GR 이전 → GR 이후)}

\section{성능 변화 내역 (GR 이후 → GR N년차)}

\section{운영특성 변화 내역 (GR 이후 → GR N년차)}

\section{시뮬레이션 결과 (상세)}

\end{document}
